\documentclass{ltxdoc}
% pro hezké formátování příkladů s výsledky
\usepackage{fontspec}
\setmainfont{Gentium}
\setmonofont{Noto Sans Mono Regular}
\setsansfont{TeX Gyre Adventor}
\usepackage{showexpl}
\usepackage{uklogo}
\usepackage{pedfstyles}
\begin{document}

\title{Balíčky pro knihovnu PedF UK}
\author{Michal Hoftich}
\date{12. 9. 2017}
\maketitle

\section{Balíček \texttt{uklogo}}

Logo knihovny. Obsahuje jeden příkaz \verb|\uklogo|

\begin{LTXexample}
  \uklogo
  \uklogo[24pt]
\end{LTXexample}

Parametr je volitelný a upřesňuje výšku loga, bez jeho použití je vysoké 32 bodů.

Je možné změnit písmo loga redefinicí příkazu \verb|\uklogofont|:

\begin{LTXexample}
  \renewfontfamily\uklogofont{Fira Sans}
  \uklogo
\end{LTXexample}

\section{Balíček \texttt{pedfstyles}}

\subsection{Titulky}
\cmd{\bigtitle}\marg{titulek}

\noindent\cmd\smalltitle\marg{titulek}

\begin{LTXexample}
  \bigtitle{Nazdar}
  \smalltitle{svete}
\end{LTXexample}

\subsection{akce}

\cmd\begin\marg{akce}

Vytvoří tabulku pro zadání akce

\cmd\datum\marg{den}


\cmd\rozsah\marg{prvni den}\marg{konecny den}

\begin{LTXexample}
  \begin{akce}
    \datum{12} & 11:00 & popisek\\
  \end{akce}
\end{LTXexample}




\end{document}
